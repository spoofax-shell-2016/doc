\chapter{Conclusion}
\label{cha:conclusion}

Over the course of the last quarter the Spoofax Language Workbench has been
explored and a read-eval-print-loop has been created that operates with any
language defined in the Spoofax Language Workbench.

\Cref{cha:introduction} gave the required background knowledge needed to
understand the problem domain. \Cref{cha:probl-defin-analys} framed the problem
definition in the context of this background. \Cref{cha:design} and
\cref{cha:implementation} discussed the design and the implementation of the
final product. \Cref{cha:evaluation} evaluated both the final product and the
process by which it came to be. \Cref{cha:discussion} reflected upon the
project, after which \cref{cha:recommendations} made recommendations to improve
the final product.

To successfully complete the project, extensive knowledge needed to be gained of both the conceptual
ideas behind programming language implementations and of the Spoofax API
implementing these concepts. This resulted in a slower pace than anticipated.

Once the required knowledge was attained, worthwhile features have been
contributed upstream. First and foremost, a properly functioning REPL,
comparable in features to those of popular programming languages. Secondly,
significant changes have been contributed to DynSem.

Despite having faced significant challenges during the start of the project,
the most important goals as set forth in the problem description have been
achieved. Therefore, the project team is still quite satisfied with the end
result: while not all requested features have been implemented, it has been shown
that it is possible to create a functioning REPL for any language defined in
Spoofax, requiring only minimal changes to the language definition.

%%% Local Variables:
%%% mode: latex
%%% TeX-master: "main"
%%% End:
