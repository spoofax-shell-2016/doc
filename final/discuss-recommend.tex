\chapter{Discussion and Recommendations}
\label{cha:disc-recomm}

In this chapter the achieved results are discussed, as well as some tasks that
have not been achieved. \Cref{cha:evaluation} already discussed how the final
product satisfied the client's requirements. This chapter goes more in-depth why
not all these requirements have been met.

Afterwards, some recommendations are made that will further enhance and provide
possible directions for future work on the product.

\section{Discussion}
\label{sec:discuss-discussion}

TODO: section introduction

\subsection{Language interpreters have to be on the classpath}
\label{sec:discuss-classpath}

\subsection{String representation of Dynsem data types}
\label{sec:discuss-dynsem-string


%\input{discuss-recommend/disc-eclipse}

\section{Recommendations}
\label{sec:discuss-future}

This section makes several recommendations to the client for future work on the
delivered product. These recommendations come forth from requirements that have
not been implemented (see \cref{ssec:eval-requirements}) and opportunities for
improvements in the delivered product. First, recommendations are made to
improve the REPL backend itself. Secondly, suggestions are made on
how to improve the Eclipse frontend. Finally, recommendations to provide
additional frontend implementations are made.

\subsection{Extending the functionality of the REPL}
\label{ssec:impr-backend}

The delivered backend is a solid product. Due to time constrains, however, some
nice to have features have not been implemented. This subsection shortly
highlights these features.

\subsection{More extensive history functionality}

The current input history implementation provides a means of iterating the
previously entered expressions in a linear way: a user can scroll back and forth
through the old entries. It would be nice if the history was searchable, too.
When implementing this, one could turn to existing command-line shells for
inspiration. The GNU readline library, for example, has two ways of searching
through the input history: via a keyboard shortcut, which when pressed allows
the user to enter a word that they remember is in the entry they are looking
for, or via an always-on setting that allows the user to enter the beginning of
the expression which in turn filters the linear iteration over the
history to only those entries starting with the entered input.

Another limitation of the current history implementation is the fact that it is
recorded per session. This means that when a user uses several languages in the
same session, the history contains entries from all those languages. Instead,
history should perhaps be kept not only per session, but also per language.

Related to the above, is implementing persistent history. The Eclipse frontend
currently only provides volatile history, which means that history is thrown
away when Eclipse is closed. The console plugin does provide persistent history,
but this implementation also leaves things to be desired: all history is saved
into the same file, resulting again in a mixed history. The recommendation that
is made here is to provide the ability to have separate, per language
persistent history. If the frontend is an IDE, these files can perhaps be saved
inside the language project's directory structure.

Improving the history implementation should result in a smoother and
faster way to interact with the REPL, enhancing the explorative nature.

\subsection{Analysis in context}

\Cref{sec:dynsem-refactor} explained how DynSem was refactored to allow passing
in an existing context in which REPL commands are evaluated. As explained
in \cref{sec:spoofax}, some languages also define rules for static analysis.
While support for evaluating an expression in a given context has been added to
DynSem, Spoofax's \textit{AnalysisService} currently does not support this
use case.

Providing analysis in the context of previous expressions is a desirable
feature that would enhance the REPL. Currently, however,
Spoofax does not provide the means to support this.

%%% Local Variables:
%%% mode: latex
%%% TeX-master: "../main"
%%% End:


\subsubsection{Adding support for alternative evaluation strategies}
\label{ssec:discuss-alternate-eval}

As discussed in \cref{sec:eval-strat}, languages developed with Spoofax are not
limited to a single interpreter. Instead, there can be different strategies for
evaluation. While this has been taken into account for the design of the
product, only a DynSem evaluation strategy has actually been implemented.

Languages with a Java backend (such as IceDust%
\footnote{https://github.com/MetaBorgCube/IceDust}) therefore do not currently
work with the REPL. To add support for specific interpreters, a language should
provide the REPL with a named implementation of the
\texttt{``IEvaluationStrategy''}. The language designer can register alternative
evaluation strategies with the REPL by extending the Guice module of the backend
and overriding the \texttt{bindEvalStrategies} method.

Finally, the evaluation strategy that will be used to evaluate input terms can
be configured via the ShellFacet ESV extensions (see \cref{sec:esv-extensions})
as illustrated in \cref{lst:eval-method}.

\begin{lstlisting}[language=esv,caption={Setting the evaluation strategy.},label={lst:eval-method}]
module editor/SL-Shell

shell
    evaluation method : "dynsem"
\end{lstlisting}



\subsection{Improvements to the Eclipse frontend}
\label{ssec:impr-eclipse}

\input{discuss-recommend/rec-eclipse}

\subsection{Implementing an IPython kernel for Jupyter notebooks}
\label{ssec:discuss-literate-programming}

\subsection{Implementing an IPython kernel for Jupyter notebooks}
\label{sec:literate-programming}

As explained in \cref{sec:a-literate-programming}, IPython (together with
Jupyter notebooks) supports both reproducible research and literate
programming. The client had expressed his interest in literate programming
functionality for Spoofax. Implementing an IPython kernel on top of the core
shell module should not take a lot of effort, however due to the problems
discussed earlier priority was given to a solid working REPL and an Eclipse
plugin. Therefore our product does not deliver an implementation of literate
programming.

IPython kernels provide a solid framework for literate programming that has
proven itself in many different languages
\footnote{https://github.com/ipython/ipython/wiki/IPython-kernels-for-other-languages}.
An IPython kernel is basically a daemon with several network sockets
representing the in and output streams of the client application. Several
message types are sent over these sockets in JSON format, such as requests to
evaluate a certain string or requests regarding the kernel
state\footnote{https://jupyter-client.readthedocs.io/en/latest/kernels.html}.

Since the invoker class from the core module just accepts any string given to
it and then resolves the command itself, implementing an IPython kernel should
be pretty straightforward. Most of the required work would be in creating an
adequate messaging framework capable of understanding all defined JSON
messages. When the messaging framework is in place user input can be sent
directly to the invoker and results can be displayed using the ``IResult''
interfaces as with any other client implementation.

\begin{figure}[htb]
  \centering
  \includegraphics[width=\textwidth]{ipython}
  \caption{A plot from data in an IPython notebook.}
  \label{fig:ipython}
\end{figure}

By implementing an IPython kernel all functionality offered by Jupyter
notebooks (see \cref{fig:ipython}) would be available to the Spoofax REPL at
once. This would directly result in the ability to live edit code interspersed
with documentation, while also allowing more complex graphical elements.

%%% Local Variables:
%%% mode: latex
%%% TeX-master: "../main"
%%% End:


%%% Local Variables:
%%% mode: latex
%%% TeX-master: "main"
%%% End:
