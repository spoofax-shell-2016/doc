\chapter{Discussion and Recommendations}
\label{cha:disc-recomm}

In this chapter the achieved results are discussed, as well as some tasks that
have not been achieved. \Cref{cha:evaluation} already discussed how the final
product satisfied the client's requirements. This chapter goes more in-depth why
not all these requirements have been met.

Afterwards, some recommendations are made that will further enhance and provide
possible directions for future work on the product.

\section{Discussion}
\label{sec:discuss-discussion}

TODO: section introduction

\subsection{Language interpreters have to be on the classpath}
\label{sec:discuss-classpath}

\subsection{String representation of Dynsem data types}
\label{sec:discuss-dynsem-string


%\section{Eclipse}
\label{sec:eclipse}

The implementation of the Eclipse plugin has been a source of exposing many
shortcomings in the initial designs. These shortcomings and how they have been
resolved are discussed in this section.

\subsection{Single- versus multithreading}
\label{ssec:threading}

The initial design assumed that the frontends and the backend would run in the
same thread. For a console based REPL, this assumption holds and greatly
simplifies the design. However, this assumption does not hold when the backend
has a frontend using a multithreaded graphical user interface toolkit. This
assumption resulted in two problems, which are listed separately in the next
sections. The solution and changes made to the design are then discussed
afterwards.

\subsubsection{Blocking- versus non-blocking input}

In a multi-threaded environment, asking graphical text entry widgets for the
entered text is rarely a blocking process. The REPL backend at the time,
however, assumed that getting the user's input was always a blocking operation.
Therefore, when a conceptual Eclipse frontend was made, the REPL spun into an
infinite loop trying to execute empty expressions.

\subsubsection{Blocking the UI thread}

Multi-threaded graphical user interface toolkits often use multiple threads. One
of these threads is designated the ``UI thread'', or user interface thread. This
thread is responsible for processing events (such as mouse clicks) and updating
the graphical representation of the widgets. All tasks that perform long running
calculations are supposed to be run in a background thread, such that the UI
thread is free to process incoming events. Instead, if a long running
computation is run in the UI thread, the widgets on the screen stop responding
to the user and the program appears to be in a frozen state.  This is exactly
what happened when the backend assumed to be run in the same thread as the
frontend: whilst the backend was evaluation expressions, Eclipse appeared to be
frozen due to this evaluation taking place in the UI thread from which the
execution was started. This issue would be even worse in case non-blocking input
were to be used.

\subsubsection{Accustoming to multi-threaded frontends}

As indicated in the previous sections, the only solution to these problems is to
allow multi-threaded frontends. 

%%% Local Variables:
%%% mode: latex
%%% TeX-master: "../main"
%%% End:


\section{Recommendations}
\label{sec:discuss-future}

This section makes several recommendations to the client for future work on the
delivered product. These recommendations come forth from requirements that have
not been implemented (see \cref{ssec:eval-requirements}). First, recommendations
are made to provide additional frontend implementations. Afterwards, alternative
evaluation strategies are discussed to have the backend work with interpreters
other than DynSem.

\subsection{Alternative Frontend Implementations}
\label{ssec:discuss-alternative-frontend}

\Cref{sec:frontends} showed that implementing different frontends is
straightforward: only a handful of interfaces have to be implemented in order to
provide a basic frontend. The client has expressed their interest in numerous
other frontends, which have not been implemented due to time contraints.
However, because providing additional implementations is straightforward,
several recommendations for additional frontends are made in this subsection.

\subsubsection{An IntelliJ Frontend}
\label{ssec:intellij}

As has been said before, an effort is currently underway to provide different
IDE-implementations of Spoofax. The Spoofax for IntelliJ IDEA
project\footnote{See:
\url{http://metaborg.org/en/latest/source/langdev/manual/env/intellij/}}
provides a plugin for the IntelliJ IDEA IDE. The client has expressed their
interest in having the REPL available in this IDE as well.

This IntelliJ IDEA REPL would then provide the same features as the current
console- and Eclipse REPLs.

%%% Local Variables:
%%% mode: latex
%%% TeX-master: "../main"
%%% End:


\subsection{Implementing an IPython kernel for Jupyter notebooks}
\label{sec:literate-programming}

As explained in \cref{sec:a-literate-programming}, IPython (together with
Jupyter notebooks) supports both reproducible research and literate
programming. The client had expressed his interest in literate programming
functionality for Spoofax. Implementing an IPython kernel on top of the core
shell module should not take a lot of effort, however due to the problems
discussed earlier priority was given to a solid working REPL and an Eclipse
plugin. Therefore our product does not deliver an implementation of literate
programming.

IPython kernels provide a solid framework for literate programming that has
proven itself in many different languages
\footnote{https://github.com/ipython/ipython/wiki/IPython-kernels-for-other-languages}.
An IPython kernel is basically a daemon with several network sockets
representing the in and output streams of the client application. Several
message types are sent over these sockets in JSON format, such as requests to
evaluate a certain string or requests regarding the kernel
state\footnote{https://jupyter-client.readthedocs.io/en/latest/kernels.html}.

Since the invoker class from the core module just accepts any string given to
it and then resolves the command itself, implementing an IPython kernel should
be pretty straightforward. Most of the required work would be in creating an
adequate messaging framework capable of understanding all defined JSON
messages. When the messaging framework is in place user input can be sent
directly to the invoker and results can be displayed using the ``IResult''
interfaces as with any other client implementation.

\begin{figure}[htb]
  \centering
  \includegraphics[width=\textwidth]{ipython}
  \caption{A plot from data in an IPython notebook.}
  \label{fig:ipython}
\end{figure}

By implementing an IPython kernel all functionality offered by Jupyter
notebooks (see \cref{fig:ipython}) would be available to the Spoofax REPL at
once. This would directly result in the ability to live edit code interspersed
with documentation, while also allowing more complex graphical elements.

%%% Local Variables:
%%% mode: latex
%%% TeX-master: "../main"
%%% End:


\subsection{Alternative Evaluation Strategies}
\label{ssec:discuss-alternate-eval}

\subsubsection{Adding support for alternative evaluation strategies}
\label{ssec:discuss-alternate-eval}

As discussed in \cref{sec:eval-strat}, languages developed with Spoofax are not
limited to a single interpreter. Instead, there can be different strategies for
evaluation. While this has been taken into account for the design of the
product, only a DynSem evaluation strategy has actually been implemented.

Languages with a Java backend (such as IceDust%
\footnote{https://github.com/MetaBorgCube/IceDust}) therefore do not currently
work with the REPL. To add support for specific interpreters, a language should
provide the REPL with a named implementation of the
\texttt{``IEvaluationStrategy''}. The language designer can register alternative
evaluation strategies with the REPL by extending the Guice module of the backend
and overriding the \texttt{bindEvalStrategies} method.

Finally, the evaluation strategy that will be used to evaluate input terms can
be configured via the ShellFacet ESV extensions (see \cref{sec:esv-extensions})
as illustrated in \cref{lst:eval-method}.

\begin{lstlisting}[language=esv,caption={Setting the evaluation strategy.},label={lst:eval-method}]
module editor/SL-Shell

shell
    evaluation method : "dynsem"
\end{lstlisting}



%%% Local Variables:
%%% mode: latex
%%% TeX-master: "main"
%%% End:
