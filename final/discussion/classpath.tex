\section{Language interpreters have to be on the classpath}
\label{sec:classpath}

In \cref{ssec:implementation} the implementation of the DynSem evaluation
strategy was outlined, together with its backend, the Truffle VM.
From the DynSem specification of a language an interpreter can be generated,
which uses Truffle's ``PolyglotEngine'' for its implementation.

As any other Truffle language, the language of the interpreter has to be
registered with Truffle by annotating it correctly\footnote{See:
http://lafo.ssw.jku.at/javadoc/truffle/latest/com/oracle/truffle/api/TruffleLanguage.Registration.html}.
DynSem generates this annotation for languages defined in Spoofax, after which
the PolyglotEngine finds and registers all annotated languages when
instantiated. However, according to the Truffle specifications the
PolyglotEngine will ``search for languages registered in the system class
loader and make them available for later evaluation\footnote{See:
http://lafo.ssw.jku.at/javadoc/truffle/latest/com/oracle/truffle/api/vm/PolyglotEngine.html}''.

As a consequence, interpreters for languages to be used inside the REPL need to
be on the classpath of the system classloader before instantiating a
PolyglotEngine, which implies that a REPL user should specify the path to
the desired language interpreter before starting the REPL. Since Truffle
enforces the use of the system classloader, attempts to load the
PolyglotEngine using a custom classloader unfortunately did not solve this
issue. We suspect this issue will become apparent in Spoofax as well as the new
Dynsem implementation is integrated more tightly.

%%% Local Variables:
%%% mode: latex
%%% TeX-master: "../main"
%%% End:
