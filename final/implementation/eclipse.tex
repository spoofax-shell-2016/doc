\section{Eclipse}
\label{sec:eclipse}

The implementation of the Eclipse plugin has been a source of exposing many
shortcomings in the initial designs. These shortcomings and how they have been
resolved are discussed in this section.

\subsection{Single- versus multithreading}
\label{ssec:threading}

The initial design assumed that the frontends and the backend would run in the
same thread. For a console based REPL, this assumption holds and greatly
simplifies the design. However, this assumption does not hold when the backend
has a frontend using a multithreaded graphical user interface toolkit. This
assumption resulted in two problems, which are listed separately in the next
sections. The solution and changes made to the design are then discussed
afterwards.

\subsubsection{Blocking- versus non-blocking input}

In a multi-threaded environment, asking graphical text entry widgets for the
entered text is rarely a blocking process. The REPL backend at the time,
however, assumed that getting the user's input was always a blocking operation.
Therefore, when a conceptual Eclipse frontend was made, the REPL spun into an
infinite loop trying to execute empty expressions.

\subsubsection{Blocking the UI thread}

Multi-threaded graphical user interface toolkits often use multiple threads. One
of these threads is designated the ``UI thread'', or user interface thread. This
thread is responsible for processing events (such as mouse clicks) and updating
the graphical representation of the widgets. All tasks that perform long running
calculations are supposed to be run in a background thread, such that the UI
thread is free to process incoming events. Instead, if a long running
computation is run in the UI thread, the widgets on the screen stop responding
to the user and the program appears to be in a frozen state.  This is exactly
what happened when the backend assumed to be run in the same thread as the
frontend: whilst the backend was evaluation expressions, Eclipse appeared to be
frozen due to this evaluation taking place in the UI thread from which the
execution was started. This issue would be even worse in case non-blocking input
were to be used.

\subsubsection{Accustoming to multi-threaded frontends}

As indicated in the previous sections, the only solution to these problems is to
allow multi-threaded frontends. 

%%% Local Variables:
%%% mode: latex
%%% TeX-master: "../main"
%%% End:
