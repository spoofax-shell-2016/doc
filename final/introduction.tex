\chapter{Introduction}
\label{cha:introduction}

Besides general purpose languages such as Java and C, nowadays there are also a
lot of domain specific languages available. Domain specific languages can
provide a lot of expressive power in a particular problem domain, thereby
significantly easing the burden on the programmer. Well known examples of such
DSLs include HTML for creating websites and SQL for dealing with relational
data.

At TU Delft significant effort has been spent to research and develop better
domain specific languages. To facilitate prototyping and creation of DSLs the
Spoofax Language Workbench was developed and first published in
2010\cite{Kats10a}. Since then Spoofax has had several releases and will soon
be arriving at version number 2.0.

The description below is taken from \href{http://spoofax.org}{spoofax.org}:
``The Spoofax Language Workbench supports the definition of all aspects of
textual languages using high-level, declarative meta-languages. From a
language definition using these meta-languages, Spoofax generates full-featured
Eclipse and IntelliJ editor plugins, as well as a command-line interface.''

To facilitate prototyping DSLs even better by adding the ability of exploratory
programming, Spoofax could be extended with a read-eval-print-loop, allowing
users to type in expressions and see the result immediately, in any DSL
supported by Spoofax. This document will detail how such a ``REPL'' has been
implemented over the course of our bachelor end project.

TODO: list structure of document

%%% Local Variables:
%%% mode: latex
%%% TeX-master: "main"
%%% End:
