\section{Problem Analysis}
\label{sec:problem-analysis}

This section gives an analysis of the problem defined in the previous section.
More specifically, this section describes the problems that are expected and
that will have to be solved while incorporating a REPL
within the larger context of Spoofax.

A language can have many different language constructs. The REPL for Spoofax,
however, should not be constrained to any particular language constructs.
Rather, it should operate on any defined language. For this reason the
implementation and particularly the interaction with the user needs to be
carefully considered.

\subsubsection{Detecting unfinished expressions for multiline editing}
\label{sec:detect-unfin-expr}
To support multiline editing, the REPL should detect that the
expression or statement is unfinished when the user presses the
``Return'' key for the next line, instead of trying to parse and
execute it. An obvious part of Spoofax that is relevant for this
problem, is the syntax definition in SDF3 (see
\cref{ssec:orgheadline1}).

\subsubsection{Language specific additional commands}
\label{sec:lang-spec-addit}
Another problem is when some additional command can be useful for a
REPL of a particular language, but would not make any sense within the
context of other languages. For example, some languages such as Python
allow for loading modules, which brings all of the definitions inside
of these modules into scope. An additional command to load a module could in that
case be useful, but would not make any sense in languages that have no
concept of definitions that can be imported.

This problem could be solved by the language designer by extending
their language with reflective capabilities. However, the language
designer might not want to extend the language with reflective
capabilities outside of the context of a REPL, for example when the
language is a DSL. It is clear therefore that a different approach
should be considered.

One possible solution is to allow for language specific configurations
that are loaded with the language definition of that language. This
can be done by extending the editor services discussed in
\cref{ssec:editor-serv} with REPL commands. Similar to menu actions,
where one can define menu buttons such as ``Run'' to run a program, one may
then even bind the ``load-file'' command to an action.

\subsubsection{Redefining terms bound to names}
\label{sec:redef-cont-bound}
When the user is prototyping methods inside the REPL, they would
likely want to be able to redefine that method to be of a different
implementation. However, this poses a similar problem as in the
previous section: for some languages it may not be possible nor
desirable to do so outside of the context of a REPL. Thus requiring
the language designer to extend the language with such abilities is
again an inadequate solution.

One could propose the same solution as in the previous section, namely to
define an additional command for redefining a class or method. Another possible
and maybe more adequate solution for redefining a term bound to a name, is to
allow the user to give the name and the new term. The name can then be used to
find the old term, so that it can be replaced with the new one.

It should be noted, however, that implementing this can slightly change the
semantics of the language when it is run inside the REPL.

%%% Local Variables:
%%% mode: latex
%%% TeX-master: "main"
%%% End:
