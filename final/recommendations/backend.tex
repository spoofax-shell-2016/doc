\subsubsection{More extensive history functionality}

The current input history implementation provides a means of iterating the
previously entered expressions in a linear way: a user can scroll back and forth
through the old entries. It would be nice if the history was searchable, too.
When implementing this, one could turn to existing command-line shells for
inspiration. The GNU Bash shell, for example, has two ways of searching through
the input history: via a keyboard shortcut, which when pressed allows the user
to enter a word that they remember is in the entry they are looking for, or via
an always-on setting that allows the user to enter the beginning of the
expression which will in turn filter the linear iteration over the history to
only those entries starting with the entered input.
Improving the history implementation should result in a smoother and faster way
to interact with the REPL, enhancing the explorative nature.

TODO: discuss central history that could be turned into history per language?
TODO: recommend missed features such as esv commands?

\subsubsection{Analysis in context}

%%% Local Variables:
%%% mode: latex
%%% TeX-master: "../main"
%%% End:
