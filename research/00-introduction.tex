\section*{Introduction}
\label{sec:introduction}
\addcontentsline{toc}{section}{Introduction}

Spoofax is a language workbench developed by TU Delft over the course of several
years. During those years it has grown to `a platform for developing textual
domain-specific languages with full-featured Eclipse editor plugins'.
Today languages in Spoofax are constructed using the meta-languages `SDF3',
`NABL', `TS' and `Stratego', where each meta-language has a specific purpose.
These meta-languages in turn are used to provide several language services,
such as syntax highlighting, code transformation and error marking.

Since Spoofax is used to develop new domain-specific languages, the ability
for rapid prototyping using a read-eval-print loop would be very convenient.
In this research report we first aim to get a better overview of the scope
of Spoofax and each of its meta-languages. Afterwards we will explore
REPLs and literate programming in order to clarify what functionality
a good implementation should have.

After having explored the domain of our project, we will give a concise
problem definition. This problem definition will then be analyzed in the context
of the services that Spoofax already provides to clarify potential problems
and their solutions in creating a REPL that works generically across all
generated languages.

At last the requirements for the product will be analyzed, as well as the
methods that will be used in order to realize these requirements.

%%% Local Variables:
%%% mode: latex
%%% TeX-master: "main"
%%% End:
