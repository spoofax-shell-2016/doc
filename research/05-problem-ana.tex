\section{Problem Analysis}
\label{sec:problem-analysis}
This section gives an analysis of the problem as defined in the
previous section \ref{sec:problem-definition}. More specifically, this
section describes the problems that are expected and to be solved
during the implementation process of the product. The first part of
this section describes the process of incorporating REPL generation
within the larger context of Spoofax. The second part analyzes the
part of the problem concerned with literate programming, but this part
will be analyzed in less detail than generating REPLs as it is unsure
whether this will be present in the final product.

\subsection{Incorporating REPL generation within Spoofax}
\label{ssec:incorp-repl-gener}
To generate solidly working REPLs for any language defined in Spoofax,
some Spoofax concepts have to be carefully considered. Since every
language can have many different language constructs, generating REPLs
should not be constrained to language specific constructs like
classes, but should rather be done generically. Therefore the
implementation and particularly the interaction with the user needs to
be thought of carefully.

This section lists the problems that are expected when trying to
implement generating REPLs generically. For each problem, their
solution approaches that should be explored during the implementation
process are given, as well as the parts of Spoofax that are relevant
for that problem.

\subsubsection{Redefining the contents bound to names}
\label{sec:redef-cont-bound}
% TODO: Is "content" the right terminology here?
\todo{FIXME: Just noting possible solutions we might want to elaborate}
We could add ...
\begin{itemize}
\item a special prefix for shell specific commands, and then...
\item a command to enumerate all possible namespaces as detected by spoofax
\item a command to switch to a namespace from this list
\item a command to list entries in the current namespace
\item expressions typed will be added to the selected namespace
\item all language agnostic, since we won't care what to call the namespace
\end{itemize}

\subsubsection{Detecting unfinished expressions for multiline editing}
\label{sec:detect-unfin-expr}

\subsection{Extending the product with literate programming}
\label{sec:extend-prod-with}

\subsection{Plug-in development in Eclipse}
\label{ssec:eclipse-plugins}

%%% Local Variables:
%%% mode: latex
%%% TeX-master: "main"
%%% End:
