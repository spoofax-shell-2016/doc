\section{Conclusion}
\label{sec:conclusion}

The purpose of this research report was to explore the problem domain, to define
and analyse the problem which the product is to resolve and to analyse its
requirements.

The first three sections explored the problem domain; the Spoofax Language
Workbench, read-eval-print loops and literate programming were investigated.
These three separate domains come together when working towards a viable product.

The next section defined the problem, which is to bring rapid prototyping and
exploratory programming to users who are defining programming languages in
Spoofax. This problem was then analyzed in the following section, wherein it
was outlined how the product will fit into the existing Spoofax codebase.
Several issues were foreseen, ... %TODO fill this in.
From the problem definition and analysis, a list of requirements were compiled.
These requirements are structured using the MoSCoW method. The ``must-haves'',
together with design goals outlined, form a minimal viable product. Finally,
the last section discussed the development frameworks and tools to be used.

With the research phase now complete, an outline as been formed as to what the
product should do and how it should be implemented.

%%% Local Variables:
%%% mode: latex
%%% TeX-master: "main"
%%% End:
