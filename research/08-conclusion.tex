\section{Conclusion}
\label{sec:conclusion}

The purpose of this research report was to explore the problem domain, to define
and analyse the problem which the product is to resolve and to analyse its
requirements.

The first three sections discussed the Spoofax Language Workbench,
read-eval-print loops and literate programming. These three domains come
together when working towards a viable product. The next section defined the
problem, which is to bring rapid prototyping and exploratory programming to
users who are defining programming languages in Spoofax. This problem was then
analyzed in the following section, wherein it was outlined how the product will
fit into the existing Spoofax codebase. Several issues were foreseen, but
because of %TODO fill this in.
solutions have not yet been proposed. Finally, the created list of requirements
was analysed. % TODO: expand.

%%% Local Variables:
%%% mode: latex
%%% TeX-master: "main"
%%% End:
